\documentclass[12pt,a4paper]{article}


% Packages
%======================================
% Page Geometry
\usepackage[DIV=14,BCOR=2mm,headinclude=true,footinclude=false]{typearea}
%======================================
% Headers and footnotes
\usepackage{fancyhdr}
%======================================
% Math packages
\usepackage{tikz}
\usepackage{physics}
\usepackage{tikz-cd}
\usepackage{mathtools}
\usepackage{amsmath, amssymb}
%======================================
% Theorem enviroments
\usepackage{amsthm}
\newtheorem{theorem}{Theorem}[section]
\newtheorem{lemma}[theorem]{Lemma}
\newtheorem{corollary}{Corollary}[theorem]
% Definition style
\theoremstyle{definition}
\newtheorem{definition}{Definition}[section]
% Remark style
\theoremstyle{remark}
\newtheorem*{remark}{Remark}
%======================================
% German ß ä ö ü
\usepackage[T1]{fontenc}
\usepackage[main=english, german]{babel}
%======================================
% Color
\usepackage{xcolor}
\definecolor{TUColor}{cmyk}{0, 1.0, 0.99, 0.40}
\definecolor{DarkBlue}{cmyk}{0.9, 0.6, 0.0, 0.6}
\definecolor{RichBlack}{cmyk}{0.75, 0.68, 0.67, 0.9}
%======================================
% Hyper links and references
\usepackage{hyperref}
\hypersetup{
    colorlinks,
    urlcolor={DarkBlue},
    citecolor={DarkBlue},
    linkcolor={RichBlack}
}
%======================================
% Citations
\usepackage[nobreak]{cite}
%======================================
% Images
\usepackage{graphicx}
\graphicspath{{./images/}}
%======================================
% For Title page
\usepackage{tabularx, booktabs}
%======================================


\begin{document}
%===========================================================
% General settings
% Font color
\color{RichBlack}
% Redefine header "fancy" style
\fancypagestyle{fancy}{
    \fancyhead{}
    \fancyfoot{}
    \fancyhead[LE,RO]{\footnotesize\itshape\nouppercase{\rightmark}}
    \fancyhead[LO,RE]{\footnotesize\itshape\nouppercase{\leftmark}}
    \fancyfoot[C]{\footnotesize\itshape\nouppercase\thepage}
    \renewcommand{\headrulewidth}{0.4pt}
    \renewcommand{\footrulewidth}{0.0pt}
}
\fancypagestyle{tocstyle}{
  \fancyhead{}
  \fancyfoot{}
  \renewcommand{\headrulewidth}{0.4pt}
  \renewcommand{\footrulewidth}{0.0pt}
}
%===========================================================


% Title page
%===========================================================
\begin{titlepage}
% Defines a new command for horizontal lines, change thickness here
\newcommand{\HRule}{\rule{\linewidth}{0.5mm}}
\begin{center}

\includegraphics[width=0.15\textwidth]{TU-Berlin-Logo.png}\\[1cm]
\begin{otherlanguage}{german}
\textsc{\LARGE Technische Universit\"at Berlin}\\[1.5cm]
\textsc{\large Fakult\"at 2}\\[0.5cm]
\textsc{\large Institut f\"ur Mathematik}\\[0.5cm]
\end{otherlanguage}

% Title
\setlength{\aboverulesep}{10pt}
\setlength{\belowrulesep}{13pt}
\begin{tabularx}{\textwidth}{ >{\centering\arraybackslash}X}
\midrule[0.5mm]
\huge\bfseries An Unnecessarily Long Thesis Title\\
\midrule[0.5mm]
\end{tabularx}

% Author and supervisor
\begin{minipage}{0.4\textwidth}
    \begin{flushleft}
        \large
        \textit{\textcolor{TUColor}{Author}}\\
        Sara Hussein \textsc{Samy}
    \end{flushleft}
\end{minipage}
~
\begin{minipage}{0.4\textwidth}
    \begin{flushright}
        \large
        \textit{\textcolor{TUColor}{Supervisor}}\\
        Dr. Jan \textsc{Techer}
    \end{flushright}
\end{minipage}

% Position the date 3/4 down the remaining page
\vspace{260 pt}
{\large\today}
\end{center}
\end{titlepage}
%===========================================================


%===========================================================
% Set Abstract page style to "fancy"
\pagestyle{fancy}
\section*{Abstract} \label{sec:abstract}
\pagebreak
%===========================================================
% Set TOC page style to "tocstyle"
\pagestyle{tocstyle}
\renewcommand{\contentsname}{Contents}
\tableofcontents
\pagebreak
%===========================================================
% Set all following pages style to "fancy"
\pagestyle{fancy}
\section{Introduction} \label{sec:introduction}
\pagebreak
%===========================================================
\section{Circular sections of quadrics}
As mentioned in the \hyperref[sec:introduction]{introduction}
\begin{align} \label{eq:Euler}
    e^{i \theta} = \cos(\theta)+i\sin(\theta)
\end{align}
The \hyperref[eq:Euler]{equation} as given is..
\pagebreak
%===========================================================
\section{Confocal quadrics}
\begin{theorem}[Pythagorean theorem]
\label{thm:pythagorean}
This is a theorem about right triangles and can be summarised in the next
equation
\[ x^2 + y^2 = z^2 \]
\end{theorem}
\begin{remark}
This statement in \autoref{thm:pythagorean} is true.
\end{remark}

\begin{definition}[Fibration]
\label{def:fibration}
A fibration is a mapping between two topological spaces that has the homotopy lifting property for every space \(X\).
\end{definition}
\pagebreak
%===========================================================
\section{Discrete confocal quadrics}
\pagebreak
%===========================================================
\section{Diagonally related nets on surfaces}
\pagebreak
%===========================================================
\section{Hyperboloids with circular cross sections}
\subsection{One-sheeted hyperboloids with circular cross sections}
\subsection{Two-sheeted hyperboloids with circular cross sections}
\pagebreak
%===========================================================
\section{A discretization of circular cross sections of hyperboloids}
\subsection{Discrete one-sheeted hyperboloids with circular cross sections}
\cite{DiscretizationConfocalQuadricsI}
\subsection{Discrete two-sheeted hyperboloids with circular cross sections}
\pagebreak
%===========================================================


%===========================================================
% Set page style to "tocstyle"
\pagestyle{tocstyle}
\addcontentsline{toc}{section}{References}
\bibliographystyle{abbrv}
\bibliography{refs}
%===========================================================


\end{document}
